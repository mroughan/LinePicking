\subsection{Right Cylinder line picking}
\label{sec:cylinder_line}

Let the random variable $W$ describe the shortest distance (measured across tbetween two points chosen randomly on the lateral surface of a right cylinder of length $L$ and diameter $D$ with $L >D$ 


\begin{align*}
f_{d}(d)=&\frac{2 \left(1-\frac{d}{L}\right)}{L}\\
f_{r}(r)=&\frac{2}{\pi  \sqrt{\text{D}^2-r^2}}\\
f_{r,d}(r,d)=&\frac{4 \left(1-\frac{d}{L}\right)}{\pi  L \sqrt{\text{D}^2-r^2}}
\end{align*}

\begin{align*}
I_{1}=&\intop_{0}^{w}
\intop_{0}^{\sqrt{w^{2}-r^{2}}}\!\!\!\!\!\!
\frac{4 \left(1-\frac{d}{L}\right)}{\pi  L \sqrt{\text{D}^2-r^2}}\,dd\,dr\\
%different factor
=&\intop_{0}^{w}
\intop_{0}^{\sqrt{w^{2}-r^{2}}}\!\!\!\!\!\!
\frac{4}{\pi  L \sqrt{\text{D}^2-r^2}} 
-\frac{4 d}{\pi  L^{2} \sqrt{\text{D}^2-r^2}}
\,dd\,dr\\
%integrate inner 
=&\intop_{0}^{w}
\left[
\frac{4 d}{\pi  L \sqrt{\text{D}^2-r^2}} 
-\frac{2 d^{2}}{\pi  L^{2} \sqrt{\text{D}^2-r^2}}
\right]_{0}^{\sqrt{w^{2}-r^{2}}}
\,dr\\
% substitute in upper and lower limits
=&
\intop_{0}^{w}\!
\frac{4 \sqrt{w^2-r^2}}{\pi  L\sqrt{\text{D}^2-r^2}}
-\frac{2 w^2}{\pi  L^2 \sqrt{\text{D}^2-r^2}}
+\frac{2 r^2}{\pi  L^2\sqrt{\text{D}^2-r^2}}
\,dr\\
%break into integrals we know how to deal with
=&
\frac{4}{\pi  L}\!\!
\intop_{0}^{w}\!
\frac{\sqrt{w^2-r^2}}{\sqrt{\text{D}^2-r^2}}
\,dr
-\frac{2 w^2}{\pi  L^2 }\!\!
\intop_{0}^{w}\!
\frac{1}{\sqrt{\text{D}^2-r^2}}
\,dr
+\frac{2}{\pi  L^2}\!\!
\intop_{0}^{w}\!
\frac{r^2}{\sqrt{\text{D}^2-r^2}}
\,dr\\
%Further manipilation to get that first integral in the form of an elliptic integral
=&
\frac{4 w}{\pi  L \text{D}}\!\!
\intop_{0}^{w}\!
\frac{\sqrt{1-\left(\frac{r}{w}\right)^2}}{\sqrt{1-\left(\frac{r}{\text{D}}\right)^2}}
% second integral use http://www.integral-table.com/ equation 33
-\frac{2 w^2}{\pi  L^2 }\left[
\arcsin\left(\frac{r}{\text{D}}\right)
\right]_{0}^{w}
% third integral use http://www.integral-table.com/ equation 37 and use arctan(x) = arcsin(XXXXX) identity  from http://en.wikipedia.org/wiki/
+\frac{1}{\pi  L^2}\left[
\text{D}^{2}\arcsin\left(\frac{r}{\text{D}}\right)
-r\sqrt{\text{D}^2-r^2}
\right]_{0}^{w}\!\\
\end{align*}

For the first integrand let $s=\frac{r}{\text{D}}$ so $ds=\frac{1}{\text{D}}\,dr$ thus 

% get the first integral in the form of an elliptic integral of the second kind
\begin{align*}
I_{1}=&
\frac{4 w}{\pi  L }\!\!
\intop_{0}^{\frac{w}{\text{D}}}\!
\frac{\sqrt{1-\left(\frac{\text{D}}{w}\right)^{2}s^{2}}}{\sqrt{1-s^2}}
\,ds
% now substitute in the upper and lower bounds in the remaining integrals
-\frac{2 w^2}{\pi  L^2 }
\arcsin\left(\frac{w}{\text{D}}\right)
+\frac{\text{D}^{2}}{\pi  L^2 }\arcsin\left(\frac{w}{\text{D}}\right)
-\frac{w}{\pi  L^2}\sqrt{\text{D}^2-w^2}\\
% first integral is in the form of an elliptic integral of the second kind see  http://mathworld.wolfram.com/EllipticIntegraloftheSecondKind.html
=&\frac{4 w}{\pi  L }E\left(\arcsin\left(\frac{w}{\text{D}}\right),\frac{\text{D}^{2}}{w^{2}}\right)
+\frac{\text{D}^{2}-2 w^2}{\pi  L^2 }\arcsin\left(\frac{w}{\text{D}}\right)
-\frac{w}{\pi  L^2}\sqrt{\text{D}^2-w^2}\\
% now using the identity at http://functions.wolfram.com/EllipticIntegrals/EllipticE2/03/01/02/0004/ we get the incomplete elliptic integral of the second type in terms of complete elliptic integrals of the first and second type
=&\frac{4 \text{D}}{\pi  L }
\left(E\left(\frac{w^2}{\text{D}^2}\right)+\left(\frac{w^2}{\text{D}^2}-1\right)K\left(\frac{\text{D}^2}{w^2}\right)\right)
% second and third term combined
+\frac{\text{D}^{2}-2 w^2}{\pi  L^2 }\arcsin\left(\frac{w}{\text{D}}\right)
-\frac{w}{\pi  L^2}\sqrt{\text{D}^2-w^2}\\
\end{align*}

Now we do the second integral

\begin{align*}
I_{2}=&\intop_{0}^{\text{D}}
\intop_{\sqrt{\text{D}^{2}-r^{2}}}^{\sqrt{w^{2}-r^{2}}}\!\!\!
\frac{4 \left(1-\frac{d}{L}\right)}{\pi  L \sqrt{\text{D}^2-r^2}}\,dd\,dr\\
%different factor
=&\intop_{0}^{\text{D}}
\intop_{\sqrt{\text{D}^{2}-r^{2}}}^{\sqrt{w^{2}-r^{2}}}\!\!\!
\frac{4}{\pi  L \sqrt{\text{D}^2-r^2}} 
-\frac{4 d}{\pi  L^{2} \sqrt{\text{D}^2-r^2}}
\,dd\,dr\\
%integrate inner 
=&\intop_{0}^{\text{D}}
\left[
\frac{4 d}{\pi  L \sqrt{\text{D}^2-r^2}} 
-\frac{2 d^{2}}{\pi  L^{2} \sqrt{\text{D}^2-r^2}}
\right]_{\sqrt{\text{D}^{2}-r^{2}}}^{\sqrt{w^{2}-r^{2}}}
\,dr\\
% substitute in lower and upper limits and do a pile of simplification
=&
\intop_{0}^{\text{D}}\!
\frac{4 \sqrt{w^2-r^2}}{\pi  L\sqrt{\text{D}^2-r^2}}
+\frac{2 \left(\text{D}^2-w^2\right)}{\pi  L^2\sqrt{\text{D}^2-r^2}}
-\frac{4}{\pi  L}
\,dr\\
% split it up into parts suitable for using integrals from tables on
=&
\frac{4}{\pi  L}
\intop_{0}^{\text{D}}\!
\frac{\sqrt{w^2-r^2}}{\sqrt{\text{D}^2-r^2}}
\,dr
+\frac{2 \left(\text{D}^2-w^2\right)}{\pi  L^2}\!\!
\intop_{0}^{\text{D}}\!
\frac{1}{\sqrt{\text{D}^2-r^2}}
\,dr
-\frac{4}{\pi  L}
\intop_{0}^{\text{D}}\!
1
\,dr\\
%Further manipilation to get that first integral in the form of an elliptic integral
=&
\frac{4 w}{\pi  L \text{D}}\!\!
\intop_{0}^{\text{D}}\!
\frac{\sqrt{1-\left(\frac{r}{w}\right)^2}}{\sqrt{1-\left(\frac{r}{\text{D}}\right)^2}}
\,dr
% second integral use http://www.integral-table.com/ equation 33
+\frac{2 \left(\text{D}^2-w^2\right)}{\pi  L^2}\left[
\arcsin\left(\frac{r}{\text{D}}\right)
\right]_{0}^{\text{D}}
% third integral no tables required
-\frac{4}{\pi  L}\left[
r
\right]_{0}^{\text{D}}
\end{align*}

For the first integrand let $s=\frac{r}{\text{D}}$ so $ds=\frac{1}{\text{D}}\,dr$ thus
% get the first integral in the form of an elliptic integral of the second kind
\begin{align*}
I_{2}=&
\frac{4 w}{\pi  L }\!\!
\intop_{0}^{1}\!
\frac{\sqrt{1-\left(\frac{\text{D}}{w}\right)^{2}s^{2}}}{\sqrt{1-s^2}}
\,ds
+\frac{\left(\text{D}^2-w^2\right)}{ L^2}
-\frac{4 \text{D}}{\pi  L}\\
=&\frac{4 w}{\pi  L }E\left(\frac{\text{D}^2}{w^2}\right)
+\frac{\left(\text{D}^2-w^2\right)}{ L^2}
-\frac{4 \text{D}}{\pi  L}
\end{align*}

And the third integral

\begin{align*}
I_{3}=&\intop_{0}^{\sqrt{w^2-L^2}}
\intop_{L}^{\sqrt{w^{2}-r^{2}}}\!\!\!
\frac{4 \left(1-\frac{d}{L}\right)}{\pi  L \sqrt{\text{D}^2-r^2}}\,dd\,dr\\
%different factor
=&\intop_{0}^{\sqrt{w^2-L^2}}
\intop_{L}^{\sqrt{w^{2}-r^{2}}}\!\!\!
\frac{4}{\pi  L \sqrt{\text{D}^2-r^2}} 
-\frac{4 d}{\pi  L^{2} \sqrt{\text{D}^2-r^2}}
\,dd\,dr\\
%integrate inner 
=&\intop_{0}^{\sqrt{w^2-L^2}}
\left[
\frac{4 d}{\pi  L \sqrt{\text{D}^2-r^2}} 
-\frac{2 d^{2}}{\pi  L^{2} \sqrt{\text{D}^2-r^2}}
\right]_{L}^{\sqrt{w^{2}-r^{2}}}
\,dr\\
% substitute in lower and upper limits and do a pile of simplification 
=&\intop_{0}^{\sqrt{w^2-L^2}}\!
\frac{4 \sqrt{w^2-r^2}}{\pi  L\sqrt{\text{D}^2-r^2}}
-\frac{2 w^2}{\pi  L^2\sqrt{\text{D}^2-r^2}}
+\frac{2 r^2}{\pi  L^2\sqrt{\text{D}^2-r^2}}
-\frac{2}{\pi  \sqrt{\text{D}^2-r^2}}
\,dr\\
% split it up into parts suitable for using integrals from tables on
=&
\frac{4}{\pi  L}\!\!\!
\intop_{0}^{\sqrt{w^2-L^2}}\!\!\!
\frac{\sqrt{w^2-r^2}}{\sqrt{\text{D}^2-r^2}}
\,dr
-\frac{2 w^2}{\pi  L^2}\!\!
\intop_{0}^{\sqrt{w^2-L^2}}\!\!\!\!\!
\frac{1}{\sqrt{\text{D}^2-r^2}}
\,dr
+\frac{2}{\pi  L^2}\!\!
\intop_{0}^{\sqrt{w^2-L^2}}\!\!\!\!\!
\frac{r^2}{\sqrt{\text{D}^2-r^2}}
\,dr
-\frac{2}{\pi}
\intop_{0}^{\sqrt{w^2-L^2}}\!\!\!\!\!
\frac{1}{\sqrt{\text{D}^2-r^2}}
\,dr\\
%Further manipilation to get that first integral in the form of an elliptic integral
=&
\frac{4 w}{\pi  L \text{D}}\!\!
\intop_{0}^{\sqrt{w^2-L^2}}\!\!\!\!\!
\frac{\sqrt{1-\left(\frac{r}{w}\right)^2}}{\sqrt{1-\left(\frac{r}{\text{D}}\right)^2}}
\,dr
% second integral use http://www.integral-table.com/ equation 33
-\frac{2 w^2}{\pi  L^2}\left[
\arcsin\left(\frac{r}{\text{D}}\right)
\right]_{0}^{\sqrt{w^2-L^2}}\\
% third integral use http://www.integral-table.com/ equation 37 and use arctan(x) = arcsin(XXXXX) identity  from http://en.wikipedia.org/wiki/
&+\frac{1}{\pi  L^2}\left[
\text{D}^{2}\arcsin\left(\frac{r}{\text{D}}\right)
-r\sqrt{\text{D}^2-r^2}
\right]_{0}^{\sqrt{w^2-L^2}}
% fourth integral use http://www.integral-table.com/ equation 33
-\frac{2}{\pi}\left[
\arcsin\left(\frac{r}{\text{D}}\right)
\right]_{0}^{\sqrt{w^2-L^2}}
\end{align*}
\\
For the first integrand let $s=\frac{r}{\text{D}}$ so $ds=\frac{1}{\text{D}}\,dr$ thus

\begin{align*}
%Further manipilation to get that first integral in the form of an elliptic integral
I_{3}=&
\frac{4 w}{\pi  L }\!\!
\intop_{0}^{\frac{\sqrt{w^2-L^2}}{\text{D}}}
\!
\frac{\sqrt{1-\left(\frac{\text{D}}{w}\right)^{2}s^{2}}}{\sqrt{1-s^2}}
\,ds
% second integral use http://www.integral-table.com/ equation 33
+\frac{\text{D}^{2}-2 w^2-2L^2}{\pi  L^2}\left[
\arcsin\left(\frac{r}{\text{D}}\right)
\right]_{0}^{\sqrt{w^2-L^2}}
% third integral 
-\frac{1}{\pi  L^2}\left[
r\sqrt{\text{D}^2-r^2}
\right]_{0}^{\sqrt{w^2-L^2}}\\
%Write the elliptical integral as E
=&\frac{4 w}{\pi  L }E\left(\arcsin\left(\frac{\sqrt{w^2-L^2}}{\text{D}}\right),\frac{\text{D}^{2}}{w^{2}}\right)
+\frac{\text{D}^{2}-2 w^2-2L^2}{\pi  L^2 }\arcsin\left(\frac{\sqrt{w^2-L^2}}{\text{D}}\right)\\
&-\frac{\sqrt{\left(w^2-L^2\right) \left(\text{D}^2+L^2-w^2\right)}}{\pi  L^2}
\\
=&-\frac{4 w \sqrt{-\left(L^2-w^2\right) \left(D^2+L^2-w^2\right)}}{\pi 
   D^2 L^2}-\frac{4 \left(3 D^4 w \sec
   ^{-1}\left(\frac{D}{\sqrt{w^2-L^2}}\right)+16 D^2 L w^2
   K\left(\frac{D^2}{w^2}\right)-16 L w^4
   K\left(\frac{D^2}{w^2}\right)-12 D^2 L^2 w \sec
   ^{-1}\left(\frac{D}{\sqrt{w^2-L^2}}\right)-16 D^2 L w^2 F\left(\csc
   ^{-1}\left(\frac{D}{\sqrt{w^2-L^2}}\right)|\frac{D^2}{w^2}\right)-16
   D^2 L w^2 E\left(\csc
   ^{-1}\left(\frac{D}{\sqrt{w^2-L^2}}\right)|\frac{D^2}{w^2}\right)+16
   L w^4 F\left(\csc
   ^{-1}\left(\frac{D}{\sqrt{w^2-L^2}}\right)|\frac{D^2}{w^2}\right)-16
   L w^4 E\left(\csc
   ^{-1}\left(\frac{D}{\sqrt{w^2-L^2}}\right)|\frac{D^2}{w^2}\right)-12
   D^2 w^3 \sec ^{-1}\left(\frac{D}{\sqrt{w^2-L^2}}\right)+16
   D^2 L w^2 E\left(\frac{D^2}{w^2}\right)+16 L w^4
   E\left(\frac{D^2}{w^2}\right)\right)}{3 \pi  D^4 L^2}-\frac{8 w
   \sqrt{-\left(L^2-w^2\right) \left(D^2+L^2-w^2\right)}}{3 \pi 
   D^4}-\frac{8 w^3 \sqrt{-\left(L^2-w^2\right)
   \left(D^2+L^2-w^2\right)}}{\pi  D^4 L^2}
\end{align*}
