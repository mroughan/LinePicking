\subsection{$n$D-Sphere-geodesic-line picking}
\label{sec:nsphere_geodesic_line}

Can we do this case in general ??? Yes we can all the way down to the circle which gives a uniform distribution.

$n$-Sphere is embedded in $\R^{n+1}$ and encloses the $n+1$-ball

\subsubsection{PDF}

Obviously, the sphere has spherical symmetry, but as a result, we can,
without loss of generality, assume the first point lies at the pole of
the sphere.

As in the $n$-Sphere above we first determine a density $f(\phi_1)$ function for $\phi_1$ then using the transform method. In fact $f(\phi_1)$ is exactly the same function as in the  $n$-Sphere thus we have:

\begin{equation}  
 f(\phi_1) = \frac{\Gamma\left(\frac{1+n}{2}\right) \sin(\phi )^{n - 1}}{\sqrt{\pi } \Gamma\left(\frac{n}{2}\right)}
\end{equation} 

We have have a function for $d$ the distance between two points in terms of $\phi_1$ so we can get  $\phi_1$ in terms of $d$ thus:

\begin{eqnarray}
  d & = & \phi_1  R.\\
\phi_1& = & \frac{d}{R}\\ 
   \frac {d \frac{d}{R}}{dd} & = & \frac{1}{R}\\
\end{eqnarray}



If $y = g(X)$ then the transform rule  is :  

\[ f_Y(y) = \left| \frac{d}{dy} \left( g^{-1}(y) \right) \right|
                f_X\left( g^{-1}(y) \right)
\]

thus 
\begin{eqnarray}
g_{R}^{n}(d) & = &  \left|\frac{1}{R}\right|  f\left(\frac{d}{R} \right)\\
 & = &\frac{\Gamma\left(\frac{1+n}{2}\right) \sin\left(\frac{d}{R}\right)^{n-1}}{\sqrt{\pi } R \Gamma\left(\frac{n}{2}\right)}
\end{eqnarray}

\subsubsection{CDF}
\begin{equation}
G_{R}^{n}(d)=\frac{1}{2}-\frac{\cos\left(\frac{d}{R}\right) \sin\left(\frac{d}{R}\right)^n \left(\sin\left(\frac{d}{R}\right)^2\right)^{-n/2}\Gamma\left(\frac{1+n}{2}\right) {}_{2}F_{1}\left(\frac{1}{2},1-\frac{n}{2},\frac{3}{2},\cos\left(\frac{d}{R}\right)^2\right) }  {\sqrt{\pi } \Gamma\left(\frac{n}{2}\right)}
\end{equation}

\subsubsection{Moments}

\begin{equation}
E[g_{R}^{n}(d)]=\frac{\pi R}{2}
\end{equation}




