\subsection{$n$D-Sphere-geodesic-line picking}
\label{sec:nsphere_geodesic_line}

Can we do this case in general ??? Yes we can all the way down to the circle which gives a uniform distribution.

$n$-Sphere is embedded in $\R^{n+1}$ and encloses the $n+1$-ball

\subsubsection{PDF}

Obviously, the sphere has spherical symmetry, but as a result, we can,
without loss of generality, assume the first point lies at the pole of
the sphere.

As in the $n$-Sphere above we first determine a density $f(\phi_1)$ function for $\phi_1$ then using the transform method.
\begin{equation}
g_{R}^{n}(t)=\frac{\Gamma\left(\frac{1+n}{2}\right) \sin\left(\frac{t}{R}\right)^{n-1}}{\sqrt{\pi } R \Gamma\left(\frac{n}{2}\right)}
\end{equation}

\subsubsection{CDF}
\begin{equation}
G_{R}^{n}(t)=\frac{1}{2}-\frac{\cos\left(\frac{t}{R}\right) \sin\left(\frac{t}{R}\right)^n \left(\sin\left(\frac{t}{R}\right)^2\right)^{-n/2}\Gamma\left(\frac{1+n}{2}\right) {}_{2}F_{1}\left(\frac{1}{2},1-\frac{n}{2},\frac{3}{2},\cos\left(\frac{t}{R}\right)^2\right) }  {\sqrt{\pi } \Gamma\left(\frac{n}{2}\right)}
\end{equation}

\subsubsection{Moments}

\begin{equation}
E[g_{R}^{n}(t)]=\frac{\pi R}{2}
\end{equation}




