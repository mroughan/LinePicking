\section{Useful definitions, functions and results}
\label{sec:useful}

This section covers some basic definitions, and results needed to make
this paper precise and self-contained.

Definitions:
\begin{description}

\item[PDF:] 

\item[CDF:]

\item[Support:]

\end{description}


Transformation of PDFs
If $y = g(X)$

\[ f_Y(y) = \left| \frac{d}{dy} \left( g^{-1}(y) \right) \right|
               f_X\left( g^{-1}(y) \right)
\]


\begin{eqnarray}
 E[ g(X) ] & = & \int y f_g (y) \, dy \\
           & = & \int g(x) f_X(x) \, dx 
\end{eqnarray}




\subsection{$\Gamma$ functions}

Definitions ...


where the Gamma function satisfies\cite[6.1.12]{Abramowitz_and_Stegun}
\begin{eqnarray}
 \label{eq:gamma}
 \Gamma(1/2) & = & \pi^{1/2}, \\
 \Gamma(3/2) & = & \frac{1}{2} \pi^{1/2}, \\
 \Gamma(k+1) & = & k!, \\
 \Gamma(k+1/2) & = & \frac{(2k)! \pi^{1/2}}{2^{2k} k!},
\end{eqnarray}



From ????
\begin{eqnarray}
  \int_0^{\pi} \sin^i(x) \, dx
    & = & \left\{ \begin{array}{ll}
        \displaystyle
      \frac{\sqrt{\pi} \Gamma\big( n/2 + 1/2 \big)}{\Gamma\big( n/2+1 \big)}.
      & \mbox{ for } i = 2k, \\
        \displaystyle
      \frac{\sqrt{\pi} \Gamma\big( n/2 + 1/2 \big)}{\Gamma\big( n/2+1 \big)}.
      & \mbox{ for } i = 2k+1, \\
           \end{array} \right. \nonumber \\
\end{eqnarray}



C standard library has routines

\subsection{$\beta$ functions}

The {\em beta function} and {\em incomplete beta function} are defined
to be 
\begin{eqnarray}
  \label{eqn:beta}
  B(p,q)    & = & \int_0^1 t^{p-1} (1 - t)^{q-1} \, dt = \frac{\Gamma(p) \Gamma(q)}{\Gamma(p+q)}, \\
  \label{eqn:beta_inc}
  B(x; p,q) & = & \int_0^x t^{p-1} (1 - t)^{q-1} \, dt.
\end{eqnarray}
From these, we define the {\em regularized beta function}
\begin{equation}
  \label{eqn:beta_reg}
   I(x; p,q) = \frac{ B(x; p,q)}{B(p,q)}.
\end{equation}

There are many results known for these functions. The few needed here
on a regular basis are given below:
\begin{eqnarray}
  I(x; p, q) & = & 1 - I(1-x; q, p) \hspace{7mm} \mbox{ from \cite{}}, \\
  I(x; p, q) & = &              
\frac{x^a (1-x)^b}{a B(a,b)}  \left\{ 
               1 +
               \sum_{i=0}^{\infty} \frac{B(a+1,i+1)}{B(a+b,i+1)} (1-x)^{i+1}
           \right\} 
     \hspace{7mm} \mbox{ from \cite[26.5.4]{Abramowitz_and_Stegun}}, \\
\end{eqnarray}
 

Beta functions need to be calculated for a number of the PDFs and CDFs
given in this work. The simple beta function can be computed using the
C-standard libraries $\Gamma$-function. There are also many libraries
which can calculate incomplete beta functions, however, in order that
this code be as self-contained as possible, we implement our own beta
function calculation based on the continued fraction approximation
described in \cite{}.


\subsection{Hypergoemetric functions}


\subsection{Elliptic integrals}



\subsection{$n$-balls and $n$-spheres}

Definitions

....


The generalized volume and surface area of an $n$-ball are given by:
\begin{eqnarray}
V_n(R) & = & C_n R^n,
 \label{eq:gen_vol} \\
S_{n-1}(R) & = & \frac{dV_n}{dR} = n C_n R^{n-1}.
 \label{eq:gen_surf}
\end{eqnarray}
where
\begin{equation}
 \label{eq:cn}
  C_n = \frac{ \pi^{n/2} }{\Gamma(n/2 + 1)}.
\end{equation}


