\subsection{$n$D-Sphere-line picking}
\label{sec:nsphere_line}

Can we do this case in general ??? yes we can !!!

$n$-Sphere is embedded in $\R^{n+1}$ and encloses the $n+1$-ball

\subsubsection{PDF}

Obviously, the sphere has spherical symmetry, but as a result, we can,
without loss of generality, assume the first point lies at the pole of
the sphere.

Points on the $n$-sphere can be parameterised by a set of angles
$\phi_1, \phi_2, \ldots, \phi_{n-1}$ where $\phi_1, \phi_2, \ldots,
\phi_{n-2}\in [0,\pi]$ and $\phi_{n-1}\in [0,2\pi)$. 


We can transform to Cartesian coordinates of $\R^{n+1}$ by
\begin{eqnarray}
  \label{eq:sphere_to_cart1}
  x_1 & = & R \cos (\phi_1 ) \\
  x_2 & = & R \sin(\phi_1) \cos (\phi_2 ) \\
  x_3 & = & R \sin(\phi_1) \sin(\phi_2) \cos (\phi_3 ) \\
      & \vdots & \\
  x_{n} & = & R \sin (\phi_1) \sin(\phi_2) \cdots \sin (\phi_{n-1}) \cos (\phi_{n})  \\
  x_{n+1} & = & R \sin (\phi_1) \sin(\phi_2) \cdots \sin (\phi_{n-1}) \sin (\phi_{n}) 
  \label{eq:sphere_to_cartn}
\end{eqnarray}

So the first point is chosen at the pole where
$\phi_1=\phi_2=\cdots=\phi_{n-1}=0$, so that ${\mathbf x} =
(R,0,0,\ldots,0)$, and the second point satisfies equations
\eqref{eq:sphere_to_cart1}-\eqref{eq:sphere_to_cartn}. 

Distance is therefore
\begin{eqnarray}
  \label{eq:d_nsphere}
  d^2 & = & R^2 \left[ 
                        (1-\cos(\phi_1))^2
                       + \sin^2 (\phi_1) \cos^2 (\phi_2 )
                       + \sin^2 (\phi_1) \sin^2 (\phi_2) \cos^2 (\phi_3 )
                       + \cdots \right.  \nonumber \\
     & &   \left.           + \sin^2 (\phi_1) \sin^2 (\phi_2) \cdots \sin^2 (\phi_{n-1}) \cos^2 (\phi_{n})
                       + \sin^2 (\phi_1) \sin^2 (\phi_2) \cdots \sin^2 (\phi_{n-1}) \sin^2 (\phi_{n}) 
                 \right].
\end{eqnarray}
The last two terms sum, using $\cos^2 (\phi_{n}) + \sin^2
(\phi_{n}) = 1$, and then the resulting last two terms sum
likewise, and so on collapsing the whole sum down to
\begin{eqnarray}
  \label{eq:d_nsphere}
  t^2 & = & R^2 \left[ 
                        (1-\cos(\phi_1))^2
                       + \sin^2 (\phi_1) \right], \nonumber \\
 & = & R^2 \left[ 1 - 2 \cos(\phi_1) + \cos^2(\phi_1)
                       + \sin^2 (\phi_1) \right], \nonumber \\
 & = & 2 R^2 \left[ 1 - \cos(\phi_1) \right].
\end{eqnarray}
So distance only depends on the first (angular) coordinate.


Uniformly chosen at random means that in any (generalized) area
element of size $t^{n}S$, there will be, on average $\lambda t^{n}S$
points (where $\lambda$ is the rate of points, which is simply the
total number of points divided by the total (generalized) area).

A small area unit on the $n$-sphere can be written (in terms of the
Jacobian $J$)
\begin{eqnarray}
  \label{eq:vn_n_sphere}
  t^nS & = & J \, d\phi_1 \, d\phi_2 \, \cdots d\phi_{n},  \nonumber \\
       & = & \sin^{n-1}(\phi_1) \sin^{n-2}(\phi_2) \cdots \sin(\phi_{n-1})
                  d\phi_1 \, d\phi_2 \, \cdots d\phi_{n}.
\end{eqnarray}
To obtain the distribution for distances, we integrate the distance
function over these elements to obtain the density of the first
angular coordinate, i.e.,
\begin{eqnarray}
  f(\phi_1)
      & = & \sin^{n-1}(\phi_1)
              \int_{0}^{2\pi} \int_{0}^{\pi} \cdots  \int_{0}^{\pi} 
              \sin^{n-2}(\phi_2) \cdots \sin(\phi_{n-1})
                   d\phi_2 \, \cdots d\phi_{n}, \nonumber \\
      & = & \sin^{n-1}(\phi_1)
              \int_{0}^{2\pi} 1 d\phi_{n} 
              \int_{0}^{\pi} \sin(\phi_{n-1}) d\phi_{n-1}  
              \cdots  
              \int_{0}^{\pi} \sin^{n-2}(\phi_2)  d\phi_{2}, \nonumber \\   
\end{eqnarray}
Now for $i>-1$, the integral % Wolfram alpha
\begin{equation}
  \label{eq:int_sin_n}
  \int_0^{\pi} \sin^i(x) \, dx = 
       \frac{\sqrt{\pi} \Gamma\big( i/2 + 1/2 \big)}{\Gamma\big( i/2+1 \big)}.
\end{equation}
Note that in each adjacent term in the product above, the Gamma
functions in the numerators and denominators cancel so the final
product collapses down to
\begin{eqnarray}
  f(\phi_1)
      & = & \sin^{n-1}(\phi_1)
              \int_{0}^{2\pi} 1 d\phi_{n} 
              \int_{0}^{\pi} \sin(\phi_{n-1}) d\phi_{n-1}  
              \cdots  
              \int_{0}^{\pi} \sin^{n-2}(\phi_2)  d\phi_{2}, \nonumber \\   
      & = & 2 \pi \pi^{(n-2)/2} \sin^{n-1}(\phi_1)  
                     \frac{\Gamma\big( 1/2 + 1/2 \big)}{\Gamma\big( (n-2)/2+1 \big)}
                 \nonumber \\   
      & = & 2 \pi^{n/2} \sin^{n-1}(\phi_1)  
                     \frac{\Gamma\big(1 \big)}{\Gamma\big( n/2 \big)}
                 \nonumber \\   
      & = &  2 \pi \sin^{n-1}(\phi_1)  
                     \frac{2 \pi^{(n-2)/2}}{\Gamma\big( (n-2)/2+1 \big)}
                 \nonumber \\   
      & = &   2 \pi C_{n-2} \sin^{n-1}(\phi_1),
\end{eqnarray}
where $C_{n-2}$ is the coefficient for the generalized volume of an
$n$-ball (see Section~\ref{sec:useful}).

In order to make $f(\phi_1)$ into a probability density function we
need to normalise it. This is achieved by dividing through by the
result of integrating $f(\phi_1)$ over $[0, \pi]$ thus.
 \begin{equation}      
 h(\phi_1) = \int_0^{\pi}  f(\phi_1) \, d\phi_1
           =  \frac{2 \pi ^{\frac{5}{2}-\frac{n}{2}} 
            \Gamma\left(\frac{n}{2}\right)^2}{\Gamma\left(\frac{1+n}{2}\right)}.
\end{equation}


Now $g(\phi_1) =\frac{f(\phi_1)}{ h(\phi_1)}$ is the  probability
density function for the distance between points on the surface of a
sphere measured by $\phi_1$, which after some simplification can be
written: 
\begin{equation}  
 g(\phi_1) = \frac{\Gamma\left(\frac{1+n}{2}\right) \sin(\phi )^{n - 1}}{\sqrt{\pi } \Gamma\left(\frac{n}{2}\right)}.
\end{equation} 
We have have a function for $t$ the distance between two points in
terms of $\phi_1$ so we can get  $\phi_1$ in terms of $t$ thus: 
\begin{eqnarray}
  t^2 & = & 2 R^2 \left[ 1 - \cos(\phi_1) \right].\\
\phi_1& =  & \cos^{-1} \left[1-\frac{t^2}{2 R^2}\right]\\ 
   \frac {d \phi_1}{dt} & = &\frac{t}{\sqrt{1-\left(1-\frac{t^2}{2 R^2}\right)^2} R^2} \\
   & = &\frac{2}{\sqrt{4 R^2 -t^2 }}.
\end{eqnarray}
If $y = g(X)$ then the transform rule  is :  
\[ f_Y(y) = \left| \frac{d}{dy} \left( g^{-1}(y) \right) \right|
                f_X\left( g^{-1}(y) \right).
\]

Thus we can write:
\begin{eqnarray}
  g^{n-{\rm sphere}}_{n,R}(t)
    & = & \left|\frac{2}{\sqrt{4 R^2 -t^2 }} \right|
             g_{\Phi}\left( \cos^{-1} \left[1-\frac{t^2}{2  R^2}\right]\right) \nonumber \\
    & = & \frac{t \left(\frac{t^2}{R^2} -\frac{t^4}{4 R^4}\right)^{\frac{1}{2} (n-2)}
             \Gamma\left[\frac{1+n}{2}\right]}{\sqrt{\pi } R^2 \Gamma\left[\frac{n}{2}\right]}  \nonumber \\
    & = & \frac{t \left(\frac{t}{R} \right)^{(n-2)}
             \left(1 -\frac{t^2}{4 R^2}\right)^{\frac{1}{2} (n-2)}
             \Gamma\left[\frac{1+n}{2}\right]}{\sqrt{\pi } R^2 \Gamma\left[\frac{n}{2}\right]}  \nonumber \\
    & = & \frac{t^{n-1}
             \left(1 -\frac{t^2}{4 R^2}\right)^{\frac{1}{2} (n-2)}
             \Gamma\left[\frac{1+n}{2}\right]}{\sqrt{\pi } R^n \Gamma\left[\frac{n}{2}\right]}  \nonumber \\
  \label{eq:nsphere_pdf_phi}
\end{eqnarray}
using the fact that $\sin\big( \cos^{-1}(x) \big) = \sqrt{1 - x^2}$.

The obvious special cases are the circle (the 1-sphere) and the 2-sphere where the
above becomes:
\begin{eqnarray}
  g^{n-{\rm sphere}}_{1,R}(t) 
   & = &\frac{\left(1 -\frac{t^2}{4 R^2}\right)^{-\frac{1}{2}}}{\pi R}, \\
  g^{n-{\rm sphere}}_{2,R}(t) 
   & = &\frac{t}{2 R^2}.
\end{eqnarray}



\subsubsection{CDF}

The cumulative density function is derived by integrating the PDF:
\begin{eqnarray}
G^{n-{\rm sphere}}_{n,R}(t)
       & = & \int_0^t g^{n-{\rm sphere}}_{n,R}(s) \, ds \nonumber \\
       & = & \frac{ \left(1-\frac{s^2}{4 R^2}\right)^{-n/2} 
                    \left( \frac{s^2}{R^2} \right)
                   -\frac{s^4}{4 R^4})^{n/2} 
                    \Gamma \left[\frac{1+n}{2}\right] {}_{2}F_{1} 
                    \left[ 1-\frac{n}{2},\frac{n}{2},1+\frac{n}{2},\frac{s^2}{4 R^2} \right]
                  }{n \sqrt{\pi } \Gamma \left[ \frac{n}{2} \right] }.
\end{eqnarray}
This can be simplified a little further using a regularised
Hypergeometric function rather than the general one above :
\begin{equation}
G^{n-{\rm sphere}}_{n,R}(t) = 
      \frac{\left(\frac{t^2}{R^2} -\frac{t^4}{4 R^4}\right)^{n/2}
            \Gamma\left[\frac{1+n}{2}\right]
             {}_{2}\bar{F}_{1}\left[1,n,\frac{2+n}{2},\frac{t^2}{4
                 R^2}\right]
           }{2 \sqrt{\pi }}.
\end{equation}


Alt derivation: for $R=1/2$, and taking $u=s^2$, so $s = \sqrt{u}$ and
$ds = du/2\sqrt{u}$
\begin{eqnarray}
G^{n-{\rm sphere}}_{n,1/2}(t)
    & = & \int_0^t g^{n-{\rm sphere}}_{n,1/2}(s) \, ds \nonumber \\
    & = & \int_0^t  2^n \frac{s^{n-1}
             \left(1 -s^2\right)^{\frac{1}{2} (n-2)}
             \Gamma\left[\frac{1+n}{2}\right]}{\sqrt{\pi }
             \Gamma\left[\frac{n}{2}\right]} 
             \, ds \nonumber \\    
    & = & \frac{ 2^n \Gamma\left[\frac{1+n}{2}\right]}{\sqrt{\pi}\Gamma\left[\frac{n}{2}\right]}
           \int_0^t s^{n-1}  \left(1 -s^2\right)^{\frac{1}{2} (n-2)}
             \, ds \nonumber \\    
    & = & \frac{ 2^n \Gamma\left[\frac{1+n}{2}\right]}{\sqrt{\pi}\Gamma\left[\frac{n}{2}\right]}
           \int_0^{t^2} u^{(n-1)/2}  \left(1 - u\right)^{\frac{1}{2} (n-2)}
             \, du/2\sqrt{u}  \nonumber \\    
    & = & \frac{ 2^{n-1} \Gamma\left[\frac{1+n}{2}\right]}{\sqrt{\pi}\Gamma\left[\frac{n}{2}\right]}
           \int_0^{t^2} u^{n/2-1}  \left(1 - u\right)^{n/2-1)}
             \, du  \nonumber \\    
    & = & \frac{ 2^{n-1} \Gamma\left[\frac{1+n}{2}\right]}{\sqrt{\pi}\Gamma\left[\frac{n}{2}\right]}
               \beta_{t^2}(n/2, n/2)  \nonumber \\    
    & = & \frac{ 2^{n-1} \Gamma\left[\frac{1+n}{2}\right]}{\sqrt{\pi}\Gamma\left[\frac{n}{2}\right]}
              \beta(n/2,n/2) I_{t^2}(n/2, n/2)  \nonumber \\    
    & = & \frac{ 2^{n-1} \Gamma\left[\frac{1+n}{2}\right]}{\sqrt{\pi}\Gamma\left[\frac{n}{2}\right]}
              \frac{\Gamma[n/2]^2}{\Gamma[n]} I_{t^2}(n/2, n/2)  \nonumber \\    
    & = & \frac{ 2^{n-1} \Gamma\left[\frac{1+n}{2}\right]\Gamma[n/2] }{\sqrt{\pi}\Gamma\left[n\right]}
              I_{t^2}(n/2, n/2)  \nonumber \\    
\end{eqnarray}




\subsubsection{Moments}


If $y = g(X)$

\[ f_Y(y) = \left| \frac{d}{dy} \left( g^{-1}(y) \right) \right|
               f_X\left( g^{-1}(y) \right)
\]


\begin{eqnarray}
 E[ g(X) ] & = & \int y f_g (y) \, dy \\
           & = & \int g(x) f_X(x) \, dx 
\end{eqnarray}

