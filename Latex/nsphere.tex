\subsection{$n$D-Sphere-line picking}
\label{sec:nsphere_line}

Can we do this case in general ???

$n$-Sphere is embedded in $\R^{n+1}$ and encloses the $n+1$-ball

\subsubsection{PDF}

Obviously, the sphere has spherical symmetry, but as a result, we can,
without loss of generality, assume the first point lies at the pole of
the sphere.

Points on the $n$-sphere can be parameterised by a set of angles
$\phi_1, \phi_2, \ldots, \phi_{n-1}$ where $\phi_1, \phi_2, \ldots,
\phi_{n-2}\in [0,\pi]$ and $\phi_{n-1}\in [0,2\pi)$. 


We can transform to Cartesian coordinates of $\R^{n+1}$ by
\begin{eqnarray}
  \label{eq:sphere_to_cart1}
  x_1 & = & R \cos (\phi_1 ) \\
  x_2 & = & R \sin(\phi_1) \cos (\phi_2 ) \\
  x_3 & = & R \sin(\phi_1) \sin(\phi_2) \cos (\phi_3 ) \\
      & \vdots & \\
  x_{n} & = & R \sin (\phi_1) \sin(\phi_2) \cdots \sin (\phi_{n-1}) \cos (\phi_{n})  \\
  x_{n+1} & = & R \sin (\phi_1) \sin(\phi_2) \cdots \sin (\phi_{n-1}) \sin (\phi_{n}) 
  \label{eq:sphere_to_cartn}
\end{eqnarray}

So the first point is chosen at the pole where
$\phi_1=\phi_2=\cdots=\phi_{n-1}=0$, so that ${\mathbf x} =
(R,0,0,\ldots,0)$, and the second point satisfies equations
\eqref{eq:sphere_to_cart1}-\eqref{eq:sphere_to_cartn}. 

Distance is therefore
\begin{eqnarray}
  \label{eq:d_nsphere}
  d^2 & = & R^2 \left[ 
                        (1-\cos(\phi_1))^2
                       + \sin^2 (\phi_1) \cos^2 (\phi_2 )
                       + \sin^2 (\phi_1) \sin^2 (\phi_2) \cos^2 (\phi_3 )
                       + \cdots \right.  \nonumber \\
     & &   \left.           + \sin^2 (\phi_1) \sin^2 (\phi_2) \cdots \sin^2 (\phi_{n-1}) \cos^2 (\phi_{n})
                       + \sin^2 (\phi_1) \sin^2 (\phi_2) \cdots \sin^2 (\phi_{n-1}) \sin^2 (\phi_{n}) 
                 \right].
\end{eqnarray}
The last two terms sum, using $\cos^2 (\phi_{n}) + \sin^2
(\phi_{n}) = 1$, and then the resulting last two terms sum
likewise, and so on collapsing the whole sum down to
\begin{eqnarray}
  \label{eq:d_nsphere}
  d^2 & = & R^2 \left[ 
                        (1-\cos(\phi_1))^2
                       + \sin^2 (\phi_1) \right], \nonumber \\
 & = & R^2 \left[ 1 - 2 \cos(\phi_1) + \cos^2(\phi_1)
                       + \sin^2 (\phi_1) \right], \nonumber \\
 & = & 2 R^2 \left[ 1 - \cos(\phi_1) \right].
\end{eqnarray}
So distance only depends on the first (angular) coordinate.


Uniformly chosen at random means that in any (generalized) area
element of size $d^{n}S$, there will be, on average $\lambda d^{n}S$
points (where $\lambda$ is the rate of points, which is simply the
total number of points divided by the total (generalized) area).

A small area unit on the $n$-sphere can be written (in terms of the
Jacobian $J$)
\begin{eqnarray}
  \label{eq:vn_n_sphere}
  d^nS & = & J \, d\phi_1 \, d\phi_2 \, \cdots d\phi_{n},  \nonumber \\
       & = & \sin^{n-1}(\phi_1) \sin^{n-2}(\phi_2) \cdots \sin(\phi_{n-1})
                  d\phi_1 \, d\phi_2 \, \cdots d\phi_{n}.
\end{eqnarray}
To obtain the distribution for distances, we integrate the distance
function over these elements to obtain the density of the first
angular coordinate, i.e.,
\begin{eqnarray}
  f(\phi_1)
      & = & \sin^{n-1}(\phi_1)
              \int_{0}^{2\pi} \int_{0}^{\pi} \cdots  \int_{0}^{\pi} 
              \sin^{n-2}(\phi_2) \cdots \sin(\phi_{n-1})
                   d\phi_2 \, \cdots d\phi_{n}, \nonumber \\
      & = & \sin^{n-1}(\phi_1)
              \int_{0}^{2\pi} 1 d\phi_{n} 
              \int_{0}^{\pi} \sin(\phi_{n-1}) d\phi_{n-1}  
              \cdots  
              \int_{0}^{\pi} \sin^{n-2}(\phi_2)  d\phi_{2}, \nonumber \\   
\end{eqnarray}
Now for $i>-1$, the integral % Wolfram alpha
\begin{equation}
  \label{eq:int_sin_n}
  \int_0^{\pi} \sin^i(x) \, dx = 
       \frac{\sqrt{\pi} \Gamma\big( i/2 + 1/2 \big)}{\Gamma\big( i/2+1 \big)}.
\end{equation}
Note that in each adjacent term in the product above, the Gamma
functions in the numerators and denominators cancel so the final
product collapses down to
\begin{eqnarray}
  f(\phi_1)
      & = & \sin^{n-1}(\phi_1)
              \int_{0}^{2\pi} 1 d\phi_{n} 
              \int_{0}^{\pi} \sin(\phi_{n-1}) d\phi_{n-1}  
              \cdots  
              \int_{0}^{\pi} \sin^{n-2}(\phi_2)  d\phi_{2}, \nonumber \\   
      & = & 2 \pi \pi^{(n-2)/2} \sin^{n-1}(\phi_1)  
                     \frac{\Gamma\big( 1/2 + 1/2 \big)}{\Gamma\big( (n-2)/2+1 \big)}
                 \nonumber \\   
      & = & 2 \pi^{n/2} \sin^{n-1}(\phi_1)  
                     \frac{\Gamma\big(1 \big)}{\Gamma\big( n/2 \big)}
                 \nonumber \\   
      & = &  2 \pi \sin^{n-1}(\phi_1)  
                     \frac{2 \pi^{(n-2)/2}}{\Gamma\big( (n-2)/2+1 \big)}
                 \nonumber \\   
      & = &   \frac{2 \pi \sin^{n-1}(\phi_1)}{C_{n-2}}
                 \nonumber \\   
\end{eqnarray}


The generalized volume and surface are of the $n$-D hyperball are
\begin{eqnarray}
 V_n(R) & = & C_n R^n,
  \label{eq:gen_vol} \\
 S_{n}(R) & = & \frac{dV_n}{dR} = n C_n R^{n-1}.
  \label{eq:gen_surf}
\end{eqnarray}
where
\begin{equation}
  \label{eq:cn}
   C_n = \frac{ \pi^{n/2} }{\Gamma(n/2 + 1)},
\end{equation}

where the Gamma function satisfies\cite[6.1.12]{Abramowitz_and_Stegun}
\begin{eqnarray}
  \label{eq:gamma}
  \Gamma(1/2) & = & \pi^{1/2}, \\
  \Gamma(3/2) & = & \frac{1}{2} \pi^{1/2}, \\
  \Gamma(k+1) & = & k!, \\
  \Gamma(k+1/2) & = & \frac{(2k)! \pi^{1/2}}{2^{2k} k!},
\end{eqnarray}
So we can write
\begin{eqnarray}
   \int_0^{\pi} \sin^i(x) \, dx
     & = & \left\{ \begin{array}{ll}
         \displaystyle
       \frac{\sqrt{\pi} \Gamma\big( n/2 + 1/2 \big)}{\Gamma\big( n/2+1 \big)}.
       & \mbox{ for } i = 2k, \\
         \displaystyle
       \frac{\sqrt{\pi} \Gamma\big( n/2 + 1/2 \big)}{\Gamma\big( n/2+1 \big)}.
       & \mbox{ for } i = 2k+1, \\
            \end{array} \right. \nonumber \\
\end{eqnarray}

We aim to have $f d\phi_1$ be uniform ????

So, say for $n=2$, we have
\begin{eqnarray}
  f(\phi_1)
      & = &   \frac{2 \pi \sin^{n-1}(\phi_1)}{C_{n-2}}
                 \nonumber \\   
      & = &   \frac{2 \pi \sin(\phi_1)}{C_{0}}
                 \nonumber \\   
      & = &   2 \pi \sin(\phi_1)  \nonumber \\   
  d^2 & = & 2 R^2 \left[ 1 - \cos(\phi_1) \right].  \nonumber \\  
\end{eqnarray}
Take 
\begin{eqnarray}
  prob(\phi_1) & = & 1/f(\phi_1) \nonumber \\  
      & = &  1/ 2 \pi \sin(\phi_1)
\end{eqnarray}


If $y = g(X)$

\[ f_Y(y) = \left| \frac{d}{dy} \left( g^{-1}(y) \right) \right|
                f_X\left( g^{-1}(y) \right)
\]


\begin{eqnarray}
  E[ g(X) ] & = & \int y f_g (y) \, dy \\
            & = & \int g(x) f_X(x) \, dx 
\end{eqnarray}


\subsubsection{CDF}


\subsubsection{Moments}
