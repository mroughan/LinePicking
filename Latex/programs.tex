\section{Programs}
\label{sec:program}

\subsection{A Rough Guide}

The code is arranged to be usable as
\begin{enumerate}

\item Directly, as a command-line function;

\item By linking into a larger set of code;

\item Called through a {\tt Matlab} MEX wrapper; or 

\item Called through a {\tt R} wrapper function.

\end{enumerate}
It is designed to be as independent of external libraries as possible,
needing only the C standard libraries. So compilation should be
straight forward on the majority of machines.

Ideally, typing {\tt make} in the top level directory should make all
of the targets, however, {\tt R} users may find it easier to install
using standard {\tt R} installation procedures (but not that these
won't necessarily construct the other components, need for instance
for Matlab).

The makefiles in the subdirectories are named {\tt gMakefile} to avoid
conflicts with the way {\tt R} interprets them, so if you wish to
remake a specific subdirectory, enter the directory and type: 
\verb|make -f gMakefile|.

There are a large number of functions defined in the code, for each of
the cases discussed above, however, there are a small set of functions
that you may need to be aware of, that allow one to call all of the
others through a simple, uniform interface.

....


\subsection{Numerical Issues}

Most of the computations in the code involve simple calculations, with
no obvious numerical issues (other than the obvious fact that floating
point arithmetic is being used).

The computations on the $n$-D ball, however, require calculation of
the incomplete beta function. We have provided a separate library to
perform this computation, but users may find they can obtain more
accurate results using third party library functions. 

...

Estimates of errors ...


\subsection{Tests}

The tools come with a set of tests to compare performance on your
system with ours, and ensure everything is working ...


